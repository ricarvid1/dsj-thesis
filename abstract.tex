% From mitthesis package
% Version: 1.01, 2023/06/19
% Documentation: https://ctan.org/pkg/mitthesis
%
% The abstract environment creates all the required headers and footnote. 
% You only need to add the text of the abstract itself.
%
% Approximately 500 words or less; try not to use formulas or special characters
% If you don't want an initial indentation, do \noindent at the start of the abstract

The field of programmable silicon photonics has garnered significant attention in recent years due to its potential to revolutionize various technological domains.
The ability to reconfigure photonic integrated circuits on demand offers unprecedented flexibility and adaptability for applications ranging from telecommunications and optical signal processing to sensing and quantum information processing.
This interest is driven by the promise of creating versatile photonic platforms that can be tailored to specific tasks through software control, much like their electronic counterparts.

This manuscript provides a comprehensive exploration of the theoretical underpinnings, the developmental journey, and the experimental validation of applications stemming from this research.
The central theme revolves around the design, implementation, and application of a multipurpose programmable photonic integrated circuit (PIC) based on a recirculating mesh architecture.
The thesis outlines the critical need for a robust technology stack, with a particular emphasis on the software infrastructure required to harness the full potential of photonic integrated circuits.
A significant portion of this work is dedicated to the in-depth documentation of various applications demonstrated on the developed photonic processor, including advancements in optical networking, tunable delay lines, optical signal processing, and topological photonics, enabled by specifically developed algorithms and high-level instructions.
Furthermore, the thesis delves into the theoretical intricacies of optical arbitrary matrix-multiply operators, proposing a novel theoretical model for Mach-Zehnder interferometers within a recirculating mesh array.
This theoretical framework is complemented by the software implementation and experimental demonstration of arbitrary parametric matrix-multiply operators on the fabricated photonic processor.
Finally, the thesis concludes by discussing the key findings, the challenges encountered and addressed, and the commercial and research prospects of the developed technology, while also outlining future directions to further the impact and accessibility of multipurpose programmable photonic platforms.
