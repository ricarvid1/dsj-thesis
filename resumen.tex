%% resumen.tex

% From mitthesis package
% Version: 1.02, 2024/06/19
% Documentation: https://ctan.org/pkg/mitthesis
% Author: David Sanchez-Jacome

\chapter*{Resumen}
\pdfbookmark[0]{Resumen}{resumen}

El campo de la fotónica de silicio programable ha atraído una gran atención en los últimos años debido a su potencial para revolucionar diversos dominios tecnológicos.
La capacidad de reconfigurar circuitos integrados fotónicos bajo demanda ofrece una flexibilidad y adaptabilidad sin precedentes para aplicaciones que abarcan desde las telecomunicaciones y el procesamiento de señales ópticas hasta la detección y el procesamiento de información cuántica.
Este interés se debe a la promesa de crear plataformas fotónicas versátiles que puedan adaptarse a tareas específicas mediante control por software, de forma similar a sus contrapartes electrónicas.

Este manuscrito ofrece una exploración exhaustiva de los fundamentos teóricos, el proceso de desarrollo y la validación experimental de las aplicaciones derivadas de esta investigación.
El tema central gira en torno al diseño, la implementación y la aplicación de un circuito integrado fotónico programable (PIC) multipropósito basado en una arquitectura de malla recirculante.
La tesis describe la necesidad crítica de una pila tecnológica robusta, con especial énfasis en la infraestructura de software necesaria para aprovechar al máximo el potencial de los circuitos integrados fotónicos.
Una parte importante de este trabajo se dedica a la documentación exhaustiva de diversas aplicaciones demostradas en el procesador fotónico desarrollado, incluyendo avances en redes ópticas, líneas de retardo ajustables, procesamiento de señales ópticas y fotónica topológica, gracias a algoritmos específicamente desarrollados e instrucciones de alto nivel.
Además, la tesis profundiza en las complejidades teóricas de los operadores ópticos de multiplicación matricial arbitraria, proponiendo un nuevo modelo teórico para interferómetros de Mach-Zehnder dentro de una matriz de malla recirculante.
Este marco teórico se complementa con la implementación de software y la demostración experimental de operadores de multiplicación matricial paramétricos arbitrarios en el procesador fotónico fabricado.
Finalmente, la tesis concluye analizando los hallazgos clave, los desafíos encontrados y abordados, y las perspectivas comerciales y de investigación de la tecnología desarrollada, a la vez que describe las futuras direcciones para impulsar el impacto y la accesibilidad de las plataformas fotónicas programables multipropósito.

\chapter*{Resum}
\pdfbookmark[0]{Resum}{resum}

El camp de la fotònica de silici programable ha atret una gran atenció en els darrers anys a causa del seu potencial per revolucionar diversos dominis tecnològics.
La capacitat de reconfigurar circuits integrats fotònics sota demanda ofereix una flexibilitat i adaptabilitat sense precedents per a aplicacions que abasten des de les telecomunicacions i el processament de senyals òptics fins a la detecció i processament d'informació quàntica.
Aquest interès es deu a la promesa de crear plataformes fotòniques versàtils que puguin adaptar-se a tasques específiques mitjançant control per programari, de manera similar a les contraparts electròniques.

Aquest manuscrit ofereix una exploració exhaustiva dels fonaments teòrics, el procés de desenvolupament i la validació experimental de les aplicacions derivades daquesta investigació.
El tema central gira al voltant del disseny, implementació i aplicació d'un circuit integrat fotònic programable (PIC) multipropòsit basat en una arquitectura de malla recirculant.
La tesi descriu la necessitat crítica d'una pila tecnològica robusta, amb un èmfasi especial en la infraestructura de programari necessària per aprofitar al màxim el potencial dels circuits integrats fotònics.
Una part important d'aquest treball es dedica a la documentació exhaustiva de diverses aplicacions demostrades al processador fotònic desenvolupat, incloent avenços en xarxes òptiques, línies de retard ajustables, processament de senyals òptics i fotònica topològica, gràcies a algorismes específicament desenvolupats i instruccions d'alt nivell.
A més, la tesi aprofundeix en les complexitats teòriques dels operadors òptics de multiplicació matricial arbitrària, proposant un nou model teòric per a interferòmetres de Mach-Zehnder dins una matriu de malla recirculant.
Aquest marc teòric es complementa amb la implementació de programari i la demostració experimental d'operadors de multiplicació matricial paramètrics arbitraris al processador fotònic fabricat.
Finalment, la tesi conclou analitzant les troballes clau, els desafiaments trobats i abordats, i les perspectives comercials i de recerca de la tecnologia desenvolupada, alhora que descriu les adreces futures per impulsar l'impacte i l'accessibilitat de les plataformes fotòniques programables multipropòsit.

