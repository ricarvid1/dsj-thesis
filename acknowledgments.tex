%% acknowledgments.tex

% From mitthesis package
% Version: 1.02, 2024/06/19
% Documentation: https://ctan.org/pkg/mitthesis
% Author: David Sanchez-Jacome

\chapter*{Acknowledgments}
\pdfbookmark[0]{Acknowledgments}{acknowledgments}

\section*{\textit{The path towards this PhD}}

My incursion into the field of integrated photonics started in Fall of 2015 with an Optoelectronics course at the University of Ottawa during an exchange semester while I was pursuing an Electrical Engineering degree from my alma mater, the Universidad San Francisco de Quito, in Ecuador.
Looking back in time, this PhD thesis started back then with me telling my former Ecuadorian classmates about 'using light waveguides instead of copper wires' in integrated chips.
As a consequence, I have to start thanking my mom for pushing me into applying for an exchange program abroad.
My exchange semester in Canada was generously funded by the Canadian Global affairs office's Emerging Leaders in the Americas Program (ELAP) scholarship.
Without it, the start of my research career would not have been possible as the resources at home were not enough to cover this endeavor.
My sincere gratitude to this program that helps LATAM students kickstart their careers in the North American country.
The professor teaching the Optoelectronics aforementioned course was Prof.~Ksenia Dolgaleva whom I'm profoundly grateful for not only inspiring me to delve into integrated optics but also for seeing something in me.
After the course was over, she invited me to do an undergraduate internship with her research group.
My formal research experience started at this point back in summer 2015.
Thank you very much Ksenia.

After my stay in Canada I felt that my optics and photonics background had many theoretical gaps, and thus I decided I wanted to do a master's in Photonics to address this problem.
Maybe it's because wanderlust is so pervasive these days among my generation that I leaned more towards pursuing this master's in Europe.
The quest for knowledge, language learning and adventure in the old continent seemed very appealing to the younger me.
Because of this, I express my sincere gratitude to the European Union (EU) Erasmus Mundus Joint Master's Degree (EMJMD) scholarship for funding my studies and expenses at the Europhotonics program.
From Fall 2017 to Fall 2019 this program enabled me to learn and travel in France, Germany and Italy and get acquainted with the broad spectrum of photonic applications.
From lenses to spectroscopy; from optical tweezers to camera sensors; from lithography machine to machine vision.
All these fields were placed in the horizon by this program.
I'm sincerely grateful to Professor Hughes Giovannini for his support when looking for internships.
Merci Beaucoup to Dr.~Irene Wang for hosting me in Grenoble, introducing me to the field of Optogenetics and helping me realize I had a strong inclination towards software development.
Grazie mille to Dr.~Salvatore Maresca and Professor Antonella Bogoni for the introduction to photonic FMCW radars, digital signal processing and their valuable tutoring and mentoring while in Pisa.
Dankeschön to Prof.~Carsten Rockstuhl for remotely supervising and tracking my Pisa research project when so very few professors were willing to oversee this project.
Without this small selfless action I would have not travelled to Pisa and, eventually, to Valencia.
Two years had passed and after different incursions into these photonics verticals I came to understand and accept that I'm more of an engineer than a scientist.
Thus, I settled on continuing my career in the optical systems field.

While searching for PhD positions and industry jobs I stumbled upon Prof.~José Capmany's work in 2020.
After contacting him, he told me about the new company he was co-founding with Dr.~Daniel Pérez López and Prof.~Ivana Gasulla Mestre.
I had very fond memories of my undergrad experience with electronic FPGAs so the concept of an Optical FPGA seemed really enticing to me.
It brought me back to the 2015's 'using optical waveguides instead of copper wires' idea with some added low-level fundamental logic on top of it to enable reconfiguration and multipurpose usage.
It seemed like the sweet spot for me and I immediately fell for the idea.
The deep tech start-up experience also seemed very appealing to me as I've always wanted to learn how to bring research to the market and contribute to it.
Another country, another adventure!
Muchas gracias Dani, José and Ivana for opening the iPronics doors to me.
Specifically, thanks a lot Dani for your direct supervision and mentoring.
You have not only taught me about programmable integrated photonics but also about entrepreneurship, people management and, more importantly, that all great things in life require perseverance and a lot of hard work.
Thank you, José, for always helping with the university bureaucracy and for providing the top-notch resources required to carry out our research activities.
Before coming to Valencia everyone who knew you told me you were "majo" (a good person).
They were right!
And we all unfortunately know this trait is seldom found in academia.
I guess it's all about trying to balance things in life.
Thank you, Ivana, for handling all the company and visa paperwork required for my arrival to Spain.
You were extremely helpful during the onboarding process and the first months.
Muchísimas gracias as well to Maria Eugenia Hernández for her help and lots of patience with visa renewals and immigration paperwork in the following years.

\section*{\textit{The team}}

During the last 4.5 years at iPronics I have moved from chip layout design to pure software and application development with incursions to the lab when needed.
This experience couldn't have been as enjoyable as it has been without the iPronics team.
I joined the company as the third employee and now the headcount is close to 50.
Working on a small multidisciplinary team to go from scratch to first-of-its-kind product can be overwhelming sometimes (most often really fun) but never boring.
There are certainly so many fond memories of almuerzos, lunch breaks and late work sessions that will go untold.
There will always be a special place in my heart for the Software and Applications team (Alejandro, Alex, Carlos, Erica, Guillermo, Lluís, María, Paco, Thomas and Vicente) I worked side-by-side during most of this period.
Special thanks to Paco and Carlos for their, mostly successful, software development evangelization and valuable lessons about life, management and computer science.
My sincere thanks to our lab manager Ana Gutierrez for her mentoring on silicon photonics good lab practices.
Many thanks as well to Ana Gonzalez for always considering me to join her on exhibition trips and making me realize the value of networking and customer engagement.
Special thanks to Luis, Juan, María and Cristina from the PIC department for their support, advice and engaging conversations when I needed a break from the software room.

\section*{\textit{The students}}

During my PhD period I sought to develop my supervision skills, and therefore I accepted as many students and interns as I could fit in my agenda.
I'm profoundly grateful with each one of them because they allowed me to improve my training and management skills and, more importantly, for teaching me many valuable lessons that I would have taken a long time to learn myself.
In particular, thanks to the great Zhenyun Xie (Vicente) who did both his master thesis project and internship on optical computing and graph-based algorithms under my tutoring.
Thanks to Dr.~Ali Cem with whom we delved into the topic of thermal crosstalk characterization and compensation on photonic chips during his PhD stay at iPronics.
Many thanks to my first ever hires, Maria and Thomas, whom I had the privilege to train and land on the company.
Our work with Maria Rodriguez Losada made possible a Python API to implement floating parametric unitary operators on the first-generation Smartlight processor.
Together with Thomas Teferi Mulugeta I have learned enormously about Datacenter interconnects and their sensitivity to phenomena such as amplification, optical crosstalk and polarization handling.
Finally, merci to Nicolas Casteleyn whose work shed light into the challenge of polarization diversity circuits in integrated photonics for datacenter interconnects.

\section*{\textit{Family and friends}}

Family and friends have been of fundamental support during this PhD.
My mom has always been supportive and has pushed me to reach my goals even when turbulent times at home have arisen.
The second major support force throughout this period has been my girlfriend, now partner, Niko.
We ventured to Europe together pursuing our professional dreams.
During these years we have faced lockdowns, moved between flats and learned to support each other's growth while keeping things at home in check.
Muchísimas gracias mi querida Niko.
Thanks to my parents and brothers for their support during these years abroad.
They have made sure I have a warm refuge every time I go back home to reload my batteries.
Les quiero mucho!

Ever since I started hopping between countries for career or education reasons I learned the hard way that friends are the family one chooses for giving and receiving support while abroad.
Muchísimas gracias Germán, Tiz, Annabell, Mich, Lau, Juanma, Cas, Darwin and Alejo for being my Valencian family.
You'll forever have a place in my heart.
Thanks for sharing with me the experience of trying to start a new life in a new country.

\section*{\textit{The ones who sculpted this manuscript}}

The writing of this manuscript was carried out during the last 6 months.
Its structure and planning received valuable feedback from Luis Torrijos and Erica Sánchez from iPronics.
During the writing process, Germán Barriga provided great help with very skillful edits to the latex structure and figures in Inkscape.
Thomas Mulugeta was kind enough to serve as a proofreader and spotted many typos and pieces of text to be improved.
Tiziana Rubio is the artist behind the front cover who was able to turn my rough ideas into a beautiful design.
Dr.~Angelina Totovič, Prof.~Mahdi Nikdast and Dr.~Daniel Pérez López served as reviewers for this manuscript.
My sincere gratitude goes to them for taking the time to read this thesis and providing me with invaluable technical insights to significantly improve the text's quality.
